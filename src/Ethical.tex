All procedures for animal research conducted in this project will follow the current legislation and in particular, Law 32/2007, of 7th November, for the care of the animals on his farm, transport, experimentation and sacrifice, in Law 6/2013, of 11th June, amending Law 32/2007, of 7 November, for the care of the animals on his farm, transport, experimentation and sacrifice and in Royal Decree 53 / 2013, of 1th February , establishing the basic standards for the protection of animals used for experimental and other scientific purposes, including teaching.

\subsubsection*{Safety regulations}
In the Experimental Radiology Area IIS La Fe properly follow biosecurity controls established by Council of Agriculture, Livestock and Fisheries of Valencia. The responsible Biological Risk and Work Hazard of IIS La Fe must make an assessment of them and obtain a User Center Registration for the project.
All personnel who could access the area should know the limitations and contraindications and comply with the safety standards.
Staff working with animals is informed of the risks inherent in the work performed, following the rules Work Hazard Prevention for workers of Experimental Radiology.
Health risks from work with experimental animals (excluding primates) are much lower than the risks from work with human patients. In any case, regulatory actions under the veterinary supervision must always be respected.

\subsubsection*{Minimising cross-contamination}

The head of the Department of Experimental Radiology, before start any exam and provide access to the animals for this study, must ensure that there are no other animal in the areas of common access.
On no account, personal, animals or materials should access from the Experimental Radiology Area to the Animal Lab without the consent of the responsible veterinarian.
It will seek to have ventilation systems for rooms with information about air changes / hour and pressure relative to the hallway and adjoining rooms.
NMR room of the Experimental Radiology Area has a positive pressure gradient from the Animal Lab to the imaging space. 

\subsubsection*{Personal}

The technician necessary for the performance of the project must have an adequate preparation with a Training Course for Animal Research Experimenter (at least Category B).

\subsubsection*{Equipment}
It is the responsibility of the principal investigator of the project along with the responsible of the Department of Experimental Radiology to ensure that the working conditions are appropriate and there is all necessary material for the study.
The image material as well as anaesthesia must be previously identified and tested. It must be check the functioning of all the devices before starting the study.
