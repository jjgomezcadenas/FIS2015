The scientists participating in this coordinated project are part of the PETALO collaboration
The purpose of the PETALO collaboration is to develop a new type of  high-sensitivity, nuclear magnetic resonance compatible, positron-electron tomography (PET) device, with enhanced Time Of Flight (TOF) sensitivity. PETALO foresees several development stages:
\begin{enumerate}
\item Construction and operation of PETALO-2, a demonstrator based in 2 LXSC. 
\item Construction and operation of PETALO-10, a small-animal PET based in 10 LXSC.
\item Construction and operation of PETALO-FULL, a PET-TOF of large acceptance and large sensitivity.
\end{enumerate}

Specifically, this project 
requires funding for the first two stages of the project. 

The PETALO collaboration is currently a three-party venture, which includes:
\begin{enumerate}
\item The Instituto de Física Corpuscular (IFIC), a join venture between University of Valencia (UV) and the Spanish Council for Research (CSIC). The group is co-led by J.J. G\'omez-Cadenas (JJGC) and J. Díaz Medina (JDM).  JJGC is the proponent of the PETALO concept and the spokesperson of NEXT. Notice that the LXSC which constitutes the keystone of PETALO uses large area SiPMs arranged in Dice Boards, which can be regarded   
as a straight-forward extension of the technology developed by the NEXT collaboration to build the NEXT-100 detector tracking plane. JDM has extensive experience in nuclear physics instrumentation and will be fully dedicated to this project. He acts as PI and overall coordinator. 
\item The Instituto de Imagen Molecular (I3M), a join venture between University Polytechnic of Valencia (UPV), the CIEMAT and the Spanish Council for Research (CSIC). The PI of the project, Vicente Herrero Bosch (VHB) is a leading electronics engineer with ample experience in ASIC design for PET applications. 
\item The  Biomedical Imaging Research Group (GIBI230) from La Fe Health Research Institute. This is the research group of the Clinical Area of Medical Imaging of La Fe Polytechnics and University Hospital. The group is lead by the prestigious radiologist Luís Martí Bonmatí (LMB), and includes a dynamic team of medical doctors, engineers and nuclear physicists.  
\end{enumerate}

The coordination of the project reflects also the structure of the collaboration which includes particle physicists with expertise in nuclear instrumentation (the IFIC group, who will be in charge of the design, construction, testing and validation of the LXSCs, and construction of PETALO-2 and PETALO-10 apparatus), electronics engineers with expertise in the development of ASICs for PET (the I3M group, who will be in charge of the front-end electronics and DAQ for PETALO), and a, interdisciplinary team of biophysicists, nuclear medicine physicists and medical doctors (the GIBI230 group, who will be in charge of quality control and safety requirements for the deployment of PETALO-10 in a clinical environment, provide access to animal farm available at the hospital la Fe, and will develop algorithms for imaging processing). In addition to the well defined sharing of responsibilities, the different teams will need to work closely, first in the commissioning of the prototypes in a non-clinical environment, then in the operation of PETALO-10 as small animal PET, and finally in the validation of PET as fully NMR compatible apparatus.  
