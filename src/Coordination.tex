The scientists participating in this coordinated project are part of the PETALO collaboration
whose purpose is to develop a new type of  high-sensitivity, nuclear magnetic resonance compatible, positron-electron tomography (PET) device, with enhanced Time Of Flight (TOF) sensitivity. PETALO is based in the Liquid Xenon Scintillating Cell (LXSC), as the basic detection unit. The project foresees several development stages:
\begin{enumerate}
\item Construction and operation of PETALO-2, a demonstrator based in 2 LXSC. 
\item Construction and operation of PETALO-10, a small-animal PET based in 10 LXSC.
\item Construction and operation of PETALO-FULL, a PET-TOF of large acceptance and large sensitivity.
\end{enumerate}

Specifically, this project 
requires funding for the first two stages of the project. 

The PETALO project is based in the technology developed by the NEXT experiment. As a result, the PETALO collaboration includes a number of physicists and engineers involved in NEXT. In addition, the project adds several experts in nuclear instrumentation, nuclear medicine, imaging, and ASIC design. Last but not least, the collaboration has a strong foundation in the medical community granted through te participation of the GIBI230 group. Indeed PETALO is currently a three-party venture including:
\begin{enumerate}
\item {\bf The Instituto de Física Corpuscular (IFIC)}, a join venture between University of Valencia (UV) and the Spanish Council for Research (CSIC). The group is co-led by J.J. G\'omez-Cadenas (JJGC) and J. Díaz Medina (JDM).  JJGC is the spokesperson of NEXT, as well as the proponent of the LXSC idea and the PETALO concept. JDM, a senior physicist with extensive experience in nuclear physics instrumentation, is the IP of the DET subproject and the coordinator of this proposal. 
\item {em The Instituto de Imagen Molecular (I3M)}, a join venture between University Polytechnic of Valencia (UPV), the CIEMAT and the Spanish Council for Research (CSIC). The group includes J.F. Toledo (JFT), Raul Esteve (RE), Vicente Herrero Bosch (VHB) and Andrés Gadea (AG). JFT and RE have designed and implemented the front-end and DAQ of the NEXT experiment. JFT is also the Project Manager of NEXT. VHB and AG (both of them co-IPs of the ASIC subproject) are leading electronics engineers with ample experience in ASIC design for PET applications.  
\item The  Biomedical Imaging Research Group (GIBI230) from La Fe Health Research Institute. This is the research group of the Clinical Area of Medical Imaging of La Fe Polytechnics and University Hospital. The group is lead by the prestigious radiologist Luís Martí Bonmatí (LMB), and includes a dynamic team of medical doctors, engineers and nuclear physicists.  
\end{enumerate}

The IFIC group will be in charge of the design, construction, testing and validation of the  PETALO-2 and PETALO-10 detectors. The I3M group will be in charge of the front-end electronics, DAQ and slow control for PETALO-2 and PETALO-10. The GIBI230 group
 will be in charge of quality control and safety requirements for the deployment of PETALO-10 in a clinical environment, will provide access to animal farm available at the hospital la Fe, and will develop algorithms for imaging processing. In addition to the well defined sharing of responsibilities, the different teams will need to work closely, first in the commissioning of the prototypes in a non-clinical environment, then in the operation of PETALO-10 as small animal PET, and finally in the validation of PETALO as fully NMR compatible apparatus.  
