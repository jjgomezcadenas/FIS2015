\paragraph{Training Capabilities of the Group}
\par The research team has already directed 5 Doctoral Thesis and is being directing 2 more at the moment. Both members of the research team have long experience in graduate and post-graduate teaching since they work as Associate Professors in the Electronics Engineering Department of "Universitat Politècnica de València". The PETALO project may become a source for Doctoral Thesis proposal which could serve as evaluation of alternative solutions for the front-end architecture. For instance a further reduction of the DAQ deadtime would be desirable in order to increase maximum event rate. The proposed solution in PETALO would increase ASIC complexity to an unaffordable level if deadtimes under 0.5 $\mu$s must be achieved. However a scheme based on deep analog FIFOs may lead to better solutions yet many issues related to noise, signal degradation etc should be first solved.
\par A usual work plan for a Doctoral Thesis in microelectronics for front-end integration should cover a three years time span as follows:
\begin{description}
  \item[TASK 1 (6 Months)] Complementary training and State of the Art review. Depending on the proposal some specific training might be needed by the student. A mandatory review of the state of the art will help to focus on the problem.
  \item[TASK 2 (2 Month)] Design Specifications. Selection of technology node and design tools.
  \item[TASK 3 (3 Months)] High level modeling of analog and digital blocks. A set of high level modelling languages will be used in order to obtain functional detailed models of the different blocks.
  \item[TASK 4 (4 Months)] Architectural Simulations based on previous models to determine design trade-offs
  \item[TASK 5 (5 Months)] Analog design. All the analog blocks will be designed and validated in a multilevel simulation testbench using digital block models.
  \item[TASK 6 (3 Months)] Digital design.
  \item[TASK 7 (4 Months)] Layout design, Parasitics Extraction and Simulation. The effects of parasitics due to layout design are included in simulations and the whole system can be simulated as a final verification process before production.
  \item[TASK 8 (1 Month)] Sign-off preparation and Foundry first iteration (Cross design rule check, bonding diagram and package selection)
  \item[TASK 9 (4 Months)] Prototype test and characterization. A testbench (including a PCB carrier) must be designed in order to test the prototype. A test protocol must be defined and a final characterization data sheet produced.
  \item[TASK 10 (4 Months)] Thesis writing and results publishing. Some of the preliminary results (especially those related to the analog design) can be published before depending on their scientific value.
\end{description}

The Principal Investigator and the members of the research group form a multidisciplinary team including physicists, nuclear physicians and telecommunications engineers inside the Biomedical Imaging Research Group GIBI230. This research group has huge experience in the training of fellows belonging to several universities from the Spanish territory and also from abroad directing thus BSc’s and MSc’s thesis as PhD thesis. Furthermore, the members of the research team have participated as teachers in several courses and seminars related to the project topic. 
The research team members have experience in training students of different backgrounds, such as Physics or Medicine and also Biomedical Engineering, being two of the members, former trainees of their Biomedical Engineering master’s thesis.
The research team and the training fellow will have enough resources and equipment available to properly develop the assigned task of the project in the Medical Imaging Clinical Area of the University and Polytechnic Hospital La Fe. 
This experience indicates that the research team has sufficient training capacity to assume the incorporation of training fellow who could finish the doctoral thesis within the period of the project. 
