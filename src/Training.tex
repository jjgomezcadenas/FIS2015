The PETALO project represents a unique opportunity for training students. 

\subsubsection*{Training Capabilities of the IFIC group}

The PI of the project has supervised  10 Ph.D. thesis during his career. The student will have the unique opportunity to work in the construction of P0 and P4, commission and analysis the data produced by the apparatus, and later participate in the imaging analysis, in particular concerning the TOF capabilities. The development of the thesis would correspond to the tasks described in the plan presented for DET. In particular, the plan of studies will include:

\begin{enumerate}
\item {\bf Study of the energy resolution using photoelectric events.} Photoelectric events are easily characterised in the LXSC, as single-site deposition events (unless conventional SSDs, the LXSC can separate multiple-site events to an expected resolution of a few mm). Then, the energy in the LXSC is measured by adding the signal of all the SiPMs. The energy resolution is obtained by fitting the photoelectric peak. The energy resolution of the LXSC6 will be measured as reference (an energy resolution of better than 3\% FWHM is expected). Then, the energy resolution of the LXSC2 will be measured. A resolution better than 5\% FWHM is expected.
\item {\bf Study of the position resolution using photoelectric events.} To estimate the position resolution from data, we will use the LXSC6 which provides 2 redundant measurements of each coordinate. Then, if $\chi_1,\chi_2$~are redundant measurements of a given coordinate, and $\Delta \chi = \chi_2 - \chi_1$~is the difference between them, the $\Delta \chi$ distribution should have mean zero and standard deviation  
$\delta \Delta \chi \sim \sqrt{2} \delta \chi$. To estimate the position resolution of LXSC2, one starts by validating the Monte Carlo simulation of the LXSC, by comparing data and Monte Carlo results for the LXSC6. After demonstrating that the Monte Carlo reproduces correctly the simulation measured with data for the LXSC6, one can use the Monte Carlo to estimate the position resolution of LXSC2. Position resolution in the range of 1-2 mm are expected. 
\item {\bf Study of the time resolution using photoelectric events.} Coincidence Resolution Time (CRT) will be measured by comparing the time stamp provided by the two opposite LXSC cells. A CRT in the range of 200-250 ps is expected.
\end{enumerate}

In addition, a program of studies for Compton interactions (starting with Compton events that deposit their full energy in the cell and show a double-site deposition, will be carried out. The program will include the study of the energy, position and time resolution for double-site Compton events, the study of the resolution to separate double-site events, and the feasibility of using Compton kinematics to improve the resolution of the back projected tracks. 


\subsubsection*{Training Capabilities of the I3M-UPV group}
\par The research team has already directed 5 Doctoral Thesis and is being directing 2 more at the moment. Both members of the research team have long experience in graduate and post-graduate teaching since they work as Associate Professors in the Electronics Engineering Department of "Universitat Politècnica de València". The PETALO project may become a source for Doctoral Thesis proposal which could serve as evaluation of alternative solutions for the front-end architecture. For instance a further reduction of the DAQ deadtime would be desirable in order to increase maximum event rate. The proposed solution in PETALO would increase ASIC complexity to an unaffordable level if deadtimes under 0.5 $\mu$s must be achieved. However a scheme based on deep analog FIFOs may lead to better solutions yet many issues related to noise, signal degradation etc should be first solved.
\par A usual work plan for a Doctoral Thesis in microelectronics for front-end integration should cover a three years time span as follows:
\begin{description}
  \item[TASK 1 (6 Months)] Complementary training and State of the Art review. Depending on the proposal some specific training might be needed by the student. A mandatory review of the state of the art will help to focus on the problem.
  \item[TASK 2 (2 Month)] Design Specifications. Selection of technology node and design tools.
  \item[TASK 3 (3 Months)] High level modeling of analog and digital blocks. A set of high level modelling languages will be used in order to obtain functional detailed models of the different blocks.
  \item[TASK 4 (4 Months)] Architectural Simulations based on previous models to determine design trade-offs
  \item[TASK 5 (5 Months)] Analog design. All the analog blocks will be designed and validated in a multilevel simulation testbench using digital block models.
  \item[TASK 6 (3 Months)] Digital design.
  \item[TASK 7 (4 Months)] Layout design, Parasitics Extraction and Simulation. The effects of parasitics due to layout design are included in simulations and the whole system can be simulated as a final verification process before production.
  \item[TASK 8 (1 Month)] Sign-off preparation and Foundry first iteration (Cross design rule check, bonding diagram and package selection)
  \item[TASK 9 (4 Months)] Prototype test and characterization. A testbench (including a PCB carrier) must be designed in order to test the prototype. A test protocol must be defined and a final characterization data sheet produced.
  \item[TASK 10 (4 Months)] Thesis writing and results publishing. Some of the preliminary results (especially those related to the analog design) can be published before depending on their scientific value.
\end{description}

\subsubsection*{Training Capabilities of the GIBI230 group}
The Principal Investigator and the members of the research group form a multidisciplinary team including physicists, nuclear physicians and telecommunications engineers inside the Biomedical Imaging Research Group GIBI230. This research group has huge experience in the training of fellows belonging to several universities from the Spanish territory and also from abroad directing thus BSc’s and MSc’s thesis as PhD thesis. Furthermore, the members of the research team have participated as teachers in several courses and seminars related to the project topic. 
The research team members have experience in training students of different backgrounds, such as Physics or Medicine and also Biomedical Engineering, being two of the members, former trainees of their Biomedical Engineering master’s thesis.
The research team and the training fellow will have enough resources and equipment available to properly develop the assigned task of the project in the Medical Imaging Clinical Area of the University and Polytechnic Hospital La Fe. 
This experience indicates that the research team has sufficient training capacity to assume the incorporation of training fellow who could finish the doctoral thesis within the period of the project. 
