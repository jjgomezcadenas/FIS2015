\paragraph{Personnel resources and requirements of the DET subproject:}

The feasibility of the DET subproject is based in the availability at IFIC of three engineers who have worked during the past five years in the NEXT experiment and have acquired the expertise and know-how needed to make feasible the construction of P2 and P10. Specifically we refer to J. Rodriguez and V. Álvarez, both of them leading electronics engineers working in the NEXT tracking plane and to Alberto Martínez, a mechanical engineer who has been the leading mechanical designer for NEXT.
These engineers have been payed up to know with funds associated to the NEXT project with exclusive dedication to NEXT. Indeed, their dedication to NEXT will be 100\% until the end of 2015. However:

\begin{enumerate}
\item The first stage of the NEXT project (the NEW detector) is currently being installed at the LSC laboratory in Canfranc. Installation will be completed by December 2015, and 2016 will be fully devoted to the operation of NEW and analysis of its data. The construction of NEXT-100 will start only when the NEW detector has validated both operation parameters (energy resolution, topological signature, electron lifetime, stability) and the background model (assessment of the radio purity of the different components of NEW, possible identification of hot spots, etc.). The NEXT project plan foresees a period of a year for this validation.
\item Consequently, the work load of the electronics and mechanics engineers in NEXT will be smaller than usual in 2016, and their time can be shared with the PETALO project.
\item Furthermore, these engineers have obtained 3-year contracts as ``support technicians" (técnicos de apoyo) to work at IFIC. They will start their contracts in later 2015 or early 2016, at which point they will no longer be payed by NEXT. As IFIC support technicians they are entitled to share their time between two projects. 
\item The bulk of the engineering effort (design, production and test of dice boards, design of cryostat and ancillary mechanics, etc) will occur during 2016. In 2017, when the work load of NEXT increases again, the corresponding work load of PETALO will be smaller. 
\item In summary, the timing of this project is ideal for the optimal utilisation of the engineering resources available at IFIC which would make it possible a smooth technology transfer from NEXT to PETALO. 
\end{enumerate}

In addition to the engineers mentioned above, and the PI (who will be fully dedicated to PETALO) the DET project requires two full-time post-docs. The first post-doc, with an instrumental profile (Post-doc DET), will work in the construction, commissioning and operation of P2 and P10. He or she will benefit from the expertise developed at IFIC during the operation of the DEMO detector from 2011 to 2015, as well as from the existing infrastructures, which include a fully operative xenon gas depuration system, which will be reused by P2 and P10. 

 The second post-doc, will be Dr. Paola Ferrario, who has played a leading role in the development of the NEXT software during the last three years. Dr. Ferrario will apply to the ``young investigators'' MINECO program for a 3 year position to work in PETALO. She will be leading the development of SIMPLE and RAP as well as the analysis of P2 and P10 data at IFIC.  
 
 Funding is requested in this project only for the post-doc with the instrumental profile. This post-doc is essential, as he or she will be the person in charge of the daily activities related with the construction of P2 and P10, including the assembly of the detectors and gas system, data taking, debugging, etc. 
 
 \paragraph{Personnel resources and requirements of the DET subproject:} 

The ASIC subproject is based in the expertise existing  at the I3M/UPV, which includes the to co-PIs, Prof. V. Herrero Bosch (VHB), Prof. Rafael Gadea (RG) on one side and professors J. F. Toledo (JFT) and R. Esteve (RE) on the other. VHB and RG are leading developers of ASIC for PET and their extensive experience qualifies them uniquely for developing a new ASIC for PETALO. On the other hand, JFT and RE have been the leading developers of the FEE and DAQ for the NEXT experiment. The solutions that PETALO has adopted for the initial operation of P2 and P10 are based in the solutions tried and tested by NEXT, as described previously. 
In addition the group has two students currently working with VHB.

The project requires an electronic engineer (Eng ASIC), who will be in charge of the daily tasks related with the acquisition, assembly and testing of the ATCA electronics used during the initial run of P2 and P10.  He or she will also develop the slow control system. The engineer will be closely supervised by JFT and RE, who have extensive experience but can devote only a fraction of their time to this project. The development of the ASIC will be entirely the responsibility of VHB, RG and their students. 