\paragraph{Personnel resources and requirements of the DET subproject:}

The feasibility of the DET subproject is based in the availability at IFIC of three engineers who have worked during the past five years in the NEXT experiment and have acquired the expertise and know-how needed to make feasible the construction of P2 and P10. Specifically we refer to J. Rodriguez and V. Álvarez, both of them leading electronics engineers working in the NEXT tracking plane and to Alberto Martínez, a mechanical engineer who has been the leading mechanical designer for NEXT.
These engineers have been payed up to know with funds associated to the NEXT project with exclusive dedication to NEXT. Indeed, their dedication to NEXT will be 100\% until the end of 2015, but the work load of the electronics and mechanics engineers in NEXT will be smaller than usual in 2016, and their time can be shared with the PETALO project. Furthermore, these engineers have obtained 3-year contracts as ``support technicians" (técnicos de apoyo) to work at IFIC. They will start their contracts in later 2015 or early 2016, at which point they will no longer be payed by NEXT. 
Moreover there is a Ph. Student, Francisco cortés who is starting a Ph. D. Thesis on simulations of the Petalo Detector, and a technician, Vanesa Delgado who has experience in cryogeny, and has worked during years with germanium detectors. 
J. Rodriguez and V. Álvarez will collaborate in the construction of the liquid xenon cells, and  V. Delgado and A. Martínez  will participate in the design and construction of the cryostat.

In summary, the timing of this project is ideal for the optimal utilisation of the engineering resources available at IFIC which would make it possible a smooth technology transfer from NEXT to PETALO. 


In addition to the engineers mentioned above, and the PI (who will be fully dedicated to PETALO) the DET project requires two full-time doctors. The first doctor, with an instrumental profile (Post-doc DET), will work in the construction, commissioning and operation of P2 and P10 and will supervise and be responsible with the PI  all the tasks of construction of cells, cryostat and gas system. He or she will benefit from the expertise developed at IFIC during the operation of the DEMO detector from 2011 to 2015, as well as from the existing infrastructures, which include a fully operative xenon gas depuration system, which will be reused by P0 and P4. 

 The second post-doc will be leading the development of SIMPLE and RAP as well as the characterisation of P0 and P4 at IFIC. She will also contribute to the imaging software analysis, in particular concerning the PET-TOF application of the device, in collaboration with Francisco Ortega.
 
This project requires funding for one post-doc during three years. We are confident that additional personnel resources can be obtained by presenting suitable candidates to he ``young investigators'' MINECO program as well as to other funding bodies (IFIC Severo Ochoa program, Juan de la Cierva, etc). In particular,  
Dr. Paola Ferrario, who has played a leading role in the development of the NEXT software during the last three years and would be the perfect candidate for the software post-doc will apply to these programs. 
 
 \paragraph{Personnel resources and requirements of the ASIC subproject:} 

The ASIC subproject is based in the expertise existing  at the I3M/UPV, which includes the to co-PIs, Prof. V. Herrero Bosch (VHB), Prof. Rafael Gadea (RG) on one side and professors J. F. Toledo (JFT) and R. Esteve (RE) on the other. VHB and RG are leading developers of ASIC for PET and their extensive experience qualifies them uniquely for developing a new ASIC for PETALO. On the other hand, JFT and RE have been the leading developers of the FEE and DAQ for the NEXT experiment. The solutions that PETALO has adopted for the initial operation of P2 and P10 are based in the solutions tried and tested by NEXT, as described previously. 
In addition the group has two students currently working with VHB.

The project requires an electronic engineer (Eng ASIC), who will be in charge of the daily tasks related with the acquisition, assembly and testing of the ATCA electronics used during the initial run of P2 and P10.  He or she will also develop the slow control system. The engineer will be closely supervised by JFT and RE, who have extensive experience but can devote only a fraction of their time to this project. The development of the ASIC will be entirely the responsibility of VHB, RG and their students. 

 \paragraph{Personnel resources and requirements of the IMG subproject:}
 The IMG subproject includes an interdisciplinary team, lead by Dr. Irene Torres, a nuclear physicist with a Ph.D. in medical imaging. 
 
 The project requires a post-doc who whose activities will be in the area of imaging requirements, reconstruction and integration of image post-processing and the extraction of imaging biomarkers.
 
The post-doc background will be focused in the field of biomedical engineering and medical physics, specially in medical imaging acquisition technologies (PET and NMR systems) and in medical imaging processing.
 
The tasks to be addressed by the Post-Doc will be under the frame of the integration of the PET solution with the NMR, the requirements of the new technology and the aspects of image reconstruction and processing. These will include:
 \begin{enumerate}
\item Definition of MRI compatibility requirements and clinical spatial resolution.
\item Study and modelling of image degradation phenomena LXSC-PET.
\item Mathematical modelling of signals from PET detectors.
\item Integration in the software of the user interface of the system.
\item Development of hybrid image reconstruction methods for PET-MRI.
\item Development of algorithms for the extraction of imaging biomarkers.
\item Integration of imaging biomarkers analysis in user interface.
\end{enumerate}
