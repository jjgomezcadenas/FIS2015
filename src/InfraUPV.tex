\subsubsection*{Subproject ASIC}
\paragraph{Available Infrastructure and Equipment (I3M-UPV)}

The I3M-UPV research team has full access through Europractice to professional ASIC design tools. Cadence Design Systems tool framework has been chosen for mixed signal microelectronics developments due to its compatibility with most of the available design kits provided by silicon foundries. It integrates both digital and analog design flows from high abstraction level descriptions (HDL code) to schematics design entry (preferred in Analog designs) and final layout integration. Along with design tools, Cadence offers high-end simulation (Spectre, AMS), physical processing and verification (ASSURA) tools. Thus the whole design flow can be covered inside the same environment using the simulation and parasitics extraction models included in the selected technology process design kit. The mandatory non-disclosure agreements (NDA) have already been signed in order to obtain access to Austria MicroSystems Foundry Services for 0.35 um and 0.18 um processes. Those are one of the most common technologies being used nowadays for the kind of ASICs being proposed in this project.

\par High performance computing hardware is available to develop the design, simulation and verification tasks associated to microelectronics developments:
\begin{itemize}
 \item 2 Fujitsu RX200S8 (64 GB de RAM, Xeon E52660V2 x2) workstations in a rack installation inside I3M computing center.
 \item 1 Fujitsu Eternus Disk Vault acting as centralized disk space server (6 TB / RAID 5).
\end{itemize}

\par I3M-UPV group can also contribute with electronics laboratory equipment in order to build a test bench for test and characterization of the sample prototypes delivered by the silicon foundry:
\begin{itemize}
 \item Keithley 2230-30-1 Programmable Low noise Power Supply with 3 individual outputs.
 \item Lecroy WaveSurfer 104MXs-B (10GS/s) 4 channel oscilloscope for high bandwidth measurements
 \item 4 LXI ZTEC Oscilloscopes (150 MHz / 16 Channels) for general test and characterization
 \item Fluke Ti125 9Hz Thermographic Camera for thermal reliability measurements.
 \item SMD soldering facilities with high resolution camera display for prototype PCB handling.
\end{itemize}

\paragraph{Co-funding proposal}
Funding is requested for:

\begin{enumerate}
\item {\bf Acquisition of the FEE and DAQ for P4}, including 22 mezzanines, 12 blades, an ATCA chasis and SFP modules for GbE:   110,304 \euro (or about 215 \euro per channel).
%\item {\bf DAQ Farm}: 20,000 \euro.
\item {\bf White Rabbit electronics}: 13,400 \euro
\item {\bf ADCs and TDCs}: 51,000 \euro.
\item {\bf Cabling, cooling and mechanics}: 12,200 \euro.
\item {\bf APE-I development}: 24,200 \euro.
\item {\bf APE-II development}: 28,400 \euro.
\item {\bf Annual fees due to Europractice membership maintenance}, which are needed to obtain the  research degree software licenses. These add up to 3,250 \euro.
\end{enumerate}

In total, co-funding of 242,754 \euro~ is requested. 
