\paragraph{Objectives of the ASIC subproject:}
The ASIC subproject is in charge of providing the front end electronics (FEE), data acquisition (DAQ) and slow controls (SC) for the P2 and P10 apparatus.  In addition the subproject will develop an ASIC for PETALO, optimised for energy resolution, fast time response, and operation at cryogenic (LXe) temperatures and high magnetic field environment.

%The FEE, DAQ and SC for the initial operation of P2 and P10 will also use the solutions which have been jointly developed between NEXT, CERN and other institutions (who joined later the project) for the CERN RD-51 Collaboration. We describe them briefly here. 
%
%\paragraph{A DAQ solution for P2/P10}
%
%We propose to use the NEXT/RD-51 DAQ system, named SRS \footnote{ref1,ref2}, which has been licensed to an European company, EicSys, and ported to the industry standard ATCA form factor.
%
%The main SRS-ATCA components are: (1) An ATCA chassis, which includes power, cooling fans and a shelf manager unit for remote monitoring and control via Ethernet; (2) ATCA blades, which are carriers with mezzanine connectors, memory, processing units (FPGAs) and I/O connectivity; and (3) mezzanine cards to adapt to a wide range of front ends. Operation with the NEXT-DEMO and NEXT-NEW detectors has shown the stable and reliable operation of SRS-ATCA. 
%
%The ADC mezzanines can achieve 60 MHz sampling rate. P2 will have one LXSC6 ($64 \times 6 = 384$~channels) and one LXSC2 ($64 \times 2 = 128$~channels) for a total of 512 channels, which require 22 mezzanines.
%
%Each ATCA blade carries two mezzanines. So, 12 ATCA blades are required, fitting in a 12-slot  ATCA chassis. Ot these, 11 blades are used for digitisers (512 channels) and 1 blade for TDCs. The ATCA blades send data to a DAQ PC farm.
%
%%DAQ cost for PETALO-2
%%====================
%%- 22 mezzanines, 1.532 €/mezz = 33.7 k€
%%- 11 blades for digitizers + 1 blade for TDC data (optional) @ 5.2 k€/blade =62.4 k€
%%- Add a 12-slot ATCA chassis @12 k€
%%- 22 SFP modules for GbE @ 100 €/ud = 2.2 k€
%%- SUBTOTAL: 110.2 k€
%
%%-GRAND DAQ SUBTOTAL: 122.2 k€ (add 15 k€ for a high performance DAQ PC farm - we need to discuss a bit tomorrow on this, as the TDC partition is also to be taken into account!!!)
%
%\paragraph{A solution to achieve high time resolution in P2/P10}
%
%One of the most promising applications of PETALO is to function as TOF-PET. This requires excellent time resolution. Fortunately, a ready solution also exist. In the near future, EicSys will be upgrading their ATCA blades to include WhiteRabbit\footnote{ref3} nodes in its I/O section. WhiteRabbit is a CERN development used to synchronise thousands of elements in a distributed DAQ and control system across kilometre distances with sub-nanosecond resolution. In small setups like a PET, synchronisation within 200 picoseconds can be achieved with a careful calibration process. WhiteRabbit is an extension to GbE over fibre optics and uses SFP transceiver modules.
%
%%A similar approach was developed by a former member of out research group at UPV, showing better performance in some aspects. White Rabbit is open hardware and so the firmware is freely available. Thus, we must invest effort (three months sound reasonable) in reducing as much as possible skew and jitter in the White Rabbit network, as this will be the main limitation to the time resolution in the PET. This implies potential enhancements in the synchronization protocols, in electronic componentes and in network architecture.
%
%%A Spanish company based in Granada, named Seven Solutions [ref4], develops and provides solutions based on WhiteRabbit. Apart from the WhiteRabbit nodes in the ATCA blades, which will be equipped by EicSys, we need the following elements from Seven Solutions to enhance the timing resolution for PETALO-2: (1) two 18-ports WhiteRabbit switches, which are a special kind of GbE switches to build the WhiteRabbit network and (2) a White Rabbit plug-in module (named WR-LEN) to provide reliable timestamps for the TDC module which will be explained in the section devoted to PETALO-2 front-end.
%
%Consequently the 512 ADC channels needed by P2 can operate across the ATCA chassis with a controlled clock phase and with a reference timestamp. An additional benefit is the availability of 512 TDC channels which will accurately provide TOF and DOI timestamped information, which in turn can be used to trigger the acquisition. Front-end boards for slow and fast paths for 64 channels, thus matching a 64-SiPM dice board, will be designed, as discussed below.
%
%%The cost of the mentioned elements is low: 3 k€ for each of the two White Rabbit switches and 600€ for each of the eight White Rabbit plug-in modules, with an additional plug-in module for development.
%%
%%Additionally, two SFP modules are required for each front-end card, making a total of sixteen (2 k€) .
%%
%%-SUBTOTAL WhiteRabbit electronics: 13.4 k€
%
%
%\paragraph{A front end for P2}
%
%The SiPMs which we plan to use in the LXSC (SENSL 6mm C/J series)
%operating inside the cryostat will have a gain of 10$^7$. According to our initial Monte Carlo simulations, the required dynamic range for P2/P10 ranges from 20 to 2000 photoelectrons.  This can result in a large signal current of several milliamps. With more than 20.000  pixel cells, the SiPM output will be extremely linear. Moreover, dark count noise will be negligible at cryostat temperatures. The only effect to deal with will be optical crosstalk between neighbouring cells. These SiPMs have and additional fast output that could be used in addition to the conventional one (anode or cathode) for the fast timing channel.
%
%The SiPM signal will have an extremely fast rising edge with a fast decay time constant of 2 ns followed by a longer (though still fast) tail with a decay time constant of 45 ns. As a result, excellent timing resolution can be achieved with this sensor to produce TOF and DIO information. We plan to amplify (current to voltage conversion with a fast operational amplifier) either the anode or the fast outputs for all SiPMs and connect the amplified signals to TDC chips, having as reference the low-jitter system clock produced by the WhiteRabbit interface. The 8-channel THS788 TDC from Texas Instruments has 13 ps resolution, with 8 ps accuracy (sigma). This would produce very accurate timestamps for the scintillation light. Take into account that 30 ps represents in liquid Xe  approx. 6,6 mm. Eight such TDCs would allow to timestamp 64 channels (one DB).
%
%At the same time, good spatial and energy resolution are required. All channels will be converted to voltage with an operational amplifier, stretched to adapt to the sampling rate and adapted to the dynamic range of the ATCA mezzanine digitising card.
%
%The slow, analog outputs will be connected via HDMI cables to the ATCA digitiser boards. The TDC outputs will be formatted in an FPGA and sent to either (1) directly to the DAQ PCs via GbE links or (2) to an ATCA blade, adapted via an LVDS digital interface developed for the NEXT detector.
%
%\paragraph{Scaling up to PETALO-10}
%
%P10 will equip twenty 64-channel dice boards. Re-using the electronics for PETALO-2 (eight dice boards), we need instrumentation for additional twelve dice boards. 
%%This means additional 1,5 times the cost of the instrumentation for PETALO-2: 300 k€.
%
%
%%As a result, PETALO-2 and PETALO-10 instrumentation costs an overall 500 k€.
%
%%The cost of the front end can be estimated as:
%%
%%"Slow" and "fast" paths for 512 channels
%%===============================
%%A very conservative estimation must be made, as the design is not yet done:
%%- Components and connectors for the slow path: approx. 30 €/channel = approx. 15.5 k€
%%- Components and connectors for the fast path: approx.  45 €/channel + 400 €/board*8boards= approx. 26.2 k€
%%- PCBs: The design could be a stacked 6U board, in order to separate analog and digital paths or could be a single board. In worst case, 16 k€ for PCB manufacturing and component mounting. This is low volume production and so cost per unit is high.
%%
%%SiPM bias units
%%=============
%%We plan to use a development from the NEXT experiment, a 16-channel programmable bias source for SiPMs with a cost of approx. 1 k€.
%%
%%- Front-end sub total is then 51 k€
%%
%%Front-end power, mechanical support and cooling
%%=======================================
%%- A cabinet worth 1.6 k€
%%- Fan tray units worth 1 k€
%%- Four HMP4040 power supplies worth 6.4 k€
%%- A 6U 19" Eurocard chassis worth 200 €
%%Power, cabinet and cooling subtotal: 9.2 k€
%%
%%Cabling
%%======
%%- 128 HDMI cables to connect the front-end to the digitizing cards
%%- optical fiber for 22 connections
%%- Subtotal cabling is approx. 3 k€
%%
%%GRAND TOTAL is 196 k€ for PETALO-2
%%
%
% 
